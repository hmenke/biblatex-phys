%% ---------------------------------------------------------------
%% biblatex-phys --- A set of biblatex implementations of
%%   physics-related bibliography styles
%% Maintained by Joseph Wright
%% E-mail: joseph.wright@morningstar2.co.uk
%% Released under the LaTeX Project Public License v1.3c or later
%% See http://www.latex-project.org/lppl.txt
%% ---------------------------------------------------------------
%%  

\documentclass[a4paper]{ltxdoc}
\usepackage[T1]{fontenc}
\usepackage[final]{microtype}
\usepackage{csquotes,lmodern}
\usepackage{hyperref}

\hypersetup{hidelinks}

\author{Joseph Wright\thanks{E-mail: 
  \href{mailto:joseph.wright@morningstar2.co.uk}
  {\texttt{joseph.wright@morningstar2.co.uk}}}}
\title{\pkg{biblatex-phys} -- A set of \pkg{biblatex} implementations of
  physics-related bibliography styles%
  \footnote{This file describes v0.0, last revised 2012/07/10.}}
\date{Released 2012/07/10}

\providecommand*{\opt}[1]{\texttt{#1}}
\providecommand*{\pkg}[1]{\textsf{#1}}

\let\DescribeOption\DescribeEnv

\RecordChanges

\begin{document}

\maketitle

\begin{abstract}
  The \pkg{biblatex-phys} bundle is a set of styles for creating bibliographies
  using \pkg{biblatex} in the style of a number common physics journals.
\end{abstract}

\section{Introduction}

The \pkg{biblatex} package introduces a completely new method for controlling
the creation of bibliographies using \BibTeX{}. This makes a great deal of
flexibility available when creating bibliographies, most of which is much more
difficult with traditional \BibTeX{} styles.

In order to use \pkg{biblatex}, an entirely new set of appropriate supporting
styles are needed. This bundle provides a number of styles for chemistry,
following the rules of some of the most important journals in the field.

\section{The styles}

The bundle currently contains zero \pkg{biblatex} style files, each of
which has its own demonstration document:
\begin{itemize}
\end{itemize}

The styles use the standard \pkg{biblatex} database requirements. This means
that a database designed for traditional \pkg{biblatex} use may need some
editing for optimal output. The accompanying example database
\texttt{biblatex-phys.bib} shows examples of all of the supported entry types
with common fields filled in.

\section{Style options}

All of the styles here add a small number of package options to the standard
set provided by \pkg{biblatex}. This allows the styles to cover the variations
seen between different journals without needing a very large number of files.

\DescribeOption{doi}
\DescribeOption{eprint}
\DescribeOption{isbn}
\DescribeOption{url}
The standard style options \opt{doi}, \opt{eprint} \opt{isbn} and
\opt{eprint}, as described in the \pkg{biblatex} manual. However, these
options are turned off as standard by the styles in the \pkg{biblatex-chem}
bundle. This reflects the fact that these entries may be present in reference
databases but are not generally included in published bibliographies. Note
that \textsc{doi} values are printed for journal articles with no pages
given, even if the \opt{doi} option is \opt{false}

\DescribeOption{subentry}
In common with the standard \pkg{biblatex} numeric styles, all of the styles
in the bundle support the boolean \texttt{subentry} option. With this set
\opt{true}, entries of type \texttt{set} are given individual labels within
the bibliography.

\DescribeOption{articletitle}
The use of article titles varies between individual journals. The
boolean option \opt{articletitle} is available is control this behaviour.
The standard settings for the \pkg{chem-acs}, \pkg{chem-angew}
and \pkg{chem-rsc} styles have this option turned off, while the
\pkg{chem-biochem} sets this option \opt{true}.

\DescribeOption{biblabel}
The format of the numbers used in the bibliography (the \enquote{bibliography
label}) varies from journal to journal even if the same general style is used.
The \opt{biblabel} option allows the user to easily set the format used. This
option takes a value from the list: \opt{parens}, \opt{brackets}, \opt{plain}
and \opt{dot}.

\DescribeOption{chaptertitle}
The option boolean \opt{chaptertitle} option is provided to allow flexibility
for the inclusion of chapter titles for \texttt{inbook} and
\texttt{incollection} entries. The standard setting is \opt{false} for all
styles in the bundle.

\DescribeOption{pageranges}
The use of full page ranges varies between journals and indeed between
different papers in individual journals. The \opt{pageranges} boolean option
is available to turn on and off printing of full page ranges, thus allowing
printing of only the first page even when the database contains the full
page range. This option is set \opt{true} as standard.

\section{New styles}

The current set of styles here is intended to form a strong base for chemists.
However, there will be the need for other styles to be created. The package
author welcomes suggestions for other styles for inclusion. It would also be
good to keep all chemistry-related \pkg{biblatex} styles in one bundle. Others
working on chemistry styles for \pkg{biblatex} are welcome to send them to the
bundle maintainer so they can be incorporated here.

\section{Errors and omissions}

Suggestions for improvement and bug reports can be logged in the package issue
database, found at
\url{https://bitbucket.org/josephwright/biblatex-phys/issues}, or can
be sent by e-mail to 
\href{mailto:joseph.wright@morningstar2.co.uk}
  {\texttt{joseph.wright@morningstar2.co.uk}}.

\PrintChanges

\end{document}

%% 
%% Copyright (C) 2012 by
%%   Joseph Wright <joseph.wright@morningstar2.co.uk>
%% 
%% It may be distributed and/or modified under the conditions of
%% the LaTeX Project Public License (LPPL), either version 1.3c of
%% this license or (at your option) any later version.  The latest
%% version of this license is in the file:
%% 
%%    http://www.latex-project.org/lppl.txt
%% 
%% This work is "maintained" (as per LPPL maintenance status) by
%%   Joseph Wright.
%% 
%% This work consists of the files biblatex-phys.bib,
%%                                 biblatex-phys.tex,
%%                                 biblatex-phys-aip.tex,
%%                                 biblatex-phys-aps.tex,
%%                                 phys-aip.bbx,
%%                                 phys-aip.cbx,
%%                                 phys-aps.bbx and
%%                                 phys-aps.cbx,
%%           and the derived files biblatex-phys.pdf
%%                                 biblatex-phys-aip.pdf and
%%                                 biblatex-phys-aps.pdf.
%%
%% End of file `biblatex-phys.tex'.